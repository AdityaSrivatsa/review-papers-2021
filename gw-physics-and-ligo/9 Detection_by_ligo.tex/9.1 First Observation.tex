\subsection{Discovery of the the First Gravitational Wave}

\hspace{0.5cm} GW150914 was detected for the first time by the two detectors of LIGO at 09:50:45 UTC on the 14th of September, 2015 \cite{PhysRevLett.116.061102}. The signal received opened up a gateway with deeper understanding of astronomy and particle physics \cite{Abbott_2016}. The source was discovered to be of a binary black hole coalescence. This detection serves to be groundbreaking in terms of both GW and binary black hole systems.

The possibility of detecting gravitational waves were feeble due to the technology available during Einstein’s theory of relativity, although, experiments in search for the signal began in the 1960s with resonant mass detectors.  The interferometers were suggested in the 60s and 70s, finally by 2000s they were set up.   
The evidence for the presence of GW was observed by Hulse and Taylor by the discovery of a binary pulsar system PSR B1913+16.1 The system depicted subsequent loss of energy. \cite{PhysRevLett.116.061102}\\

\subsubsection{Source of GW150914}
\hspace{0.5cm} The source of GW150914 is a binary black hole merger. On analysis it was theorized that the two black holes were an undisturbed binary star system whose approximate masses were 36 and 29 solar masses successfully collapsing into a single black hole \cite{Abbott_2016} . Studies suggested that the mass of the system decreased considerably after the merger, indicating the emission of gravitational waves \cite{Ligo_org},\cite{LIGO_org}. From the merger, energy with three times the mass of our sun was converted into gravitational wave energy \cite{LIGO_org}.

This system is located 1.3 billion light years away from our solar system. The coalescence produced tremendous power and energy during the final 20 milliseconds of the merger. The increase in their tangential velocity to 60 percent the speed of light, the short separation of 350km between them, orbital frequency of 75 Hz, half the gravitational frequency of 150 Hz, confirms the signal to be from a merger of two enormous black holes because no other compact objects other than black holes can come that close without merging, not even neutron stars as they wouldn't have the required mass. \cite{PhysRevLett.116.061102},\cite{LIGO_org}\\

\subsubsection{Detection of GW150914}
\hspace{0.5cm} The two LIGO detectors at Washington State and Louisiana received the GW150914 signal, however, they were running in engineering mode. Hence, it required a 16 day analysis to confirm the signal to be legitimate and not a test simulation \cite{LIGO_org}. In order to confirm its validity, the environment detectors were checked to have no disturbances having similar properties as the GW150914 signal. At the time, LIGO was the only observing detector, the Virgo detector was not functional since it was being upgraded while GEO 600 was not sensitive enough to catch the signal \cite{PhysRevLett.116.061102}.

The LIGO detector at Hanford suffered a 7 millisecond delay than Livingston. The signal was processed in only 3 minutes after detection. It lasted for 0.2 seconds during which its frequency increased in 8 cycles from 35 Hz to 150 Hz. By Signal conversion process, when the signal was converted, it was in the audible range and created a noise similar to the chirp of a bird and was termed as the chirp signal \cite{PhysRevLett.116.061102},\cite{LIGO_org}.  

The LIGO detectors had successfully detected the gravitational wave signal emitted from a binary black hole system in 2015. It was a successful prediction of the general theory of relativity. The observations served influential in terms of both the signal as well as existence of binary black hole mergers \cite{PhysRevLett.116.061102}.

\pagebreak
















































































\pagebreak