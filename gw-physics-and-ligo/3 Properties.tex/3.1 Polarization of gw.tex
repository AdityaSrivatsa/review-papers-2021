\subsection{Polarization of Gravitational waves}


Gravitational waves can also be. Since they are three dimensional waves their polarization can be restricted to two forms where the the amplitude tensor $A^{\mu\nu}$ has two forms $A^{\mu\nu}_{+}$ and $A^{\mu\nu}_{\times}$ which are orthogonal to each other \cite{Dirkes_2018}. They can be represented as 

\begin{equation}
    A^{\mu\nu}_{+} = h_{+}\, \varepsilon^{\mu\nu}_{+}
\end{equation}

\begin{equation}
    A^{\mu\nu}_{\times} = h_{\times} \,\varepsilon^{\mu\nu}_{\times}
\end{equation}

\noindent where $\varepsilon^{\mu\nu}_{+}$ and $\varepsilon^{\mu\nu}_{\times}$ are unit polarization tensors.

\begin{equation}
\varepsilon^{\mu\nu}_{+} =
\begin{bmatrix}
0 & 0 & 0 & 0 \\
0 & +1 & 0 & 0 \\
0 & 0 & -1 & 0 \\
0 & 0 & 0 & 0 \\
\end{bmatrix}
\end{equation}
\\
\begin{equation}
\varepsilon^{\mu\nu}_{\times} =
\begin{bmatrix}
0 & 0 & 0 & 0 \\
0 & 0 & +1 & 0 \\
0 & +1 & 0 & 0 \\
0 & 0 & 0 & 0 \\
\end{bmatrix}
\end{equation}

\noindent In general relativity any tensor with indices $\mu\nu$ is a rank 2 tensor with 4 rows and 4 columns where each index can take values of space time coordinates which are $(t,x,y,z)$ , and position of each element is associated with any two coordinates. Thus in such tensors, the positions of elements are associated with space-time as follows:

\begin{equation*}
    \begin{bmatrix}
    tt & tx & ty & tz \\
    xt & xx & xy & xz \\
    yt & yx & yy & yz \\
    zt & zx & zy & zz \\
    \end{bmatrix}
\end{equation*}

\noindent So when we compare the unit polarization tensors $\varepsilon^{\mu\nu}_{+}$ and $\varepsilon^{\mu\nu}_{\times}$ with the above one, we see that in $\varepsilon^{\mu\nu}_{+}$ the non zero entries are +1 in $`xx$' direction and -1 in $`yy$' direction, hence the $A^{\mu\nu}_{+}$ amplitude is oriented only along X and Y axes, thus this gravitational wave which oscillates along X and Y axes is called as `PLUS' polarized wave because the vibration resembles `+' symbol. But in $\varepsilon^{\mu\nu}_{\times}$ the non zero entries are +1 in $`xy$' direction and -1 in $`yx$' direction, hence the $A^{\mu\nu}_{+}$ amplitude is oriented in the `XY' plane at a an angle of 45$\degree$ to the axes, thus this gravitational wave which oscillates in the `XY' plane at a an angle of 45$\degree$ to the axes is called as `CROSS' polarized wave because the vibration resembles `$\times$' symbol. 
\\

\noindent So the equation of polarized gravitational waves are:-\\
(+) wave $\Rightarrow $  $\tilde{h}_{\mu\nu} = h_{+}\, \varepsilon^{\mu\nu}_{+}\, e^{i(\omega t - k_{z}z)}$\\
$(\times)$ wave $\Rightarrow $  $\tilde{h}_{\mu\nu} = h_{\times}\, \varepsilon^{\mu\nu}_{\times}\, e^{i(\omega t - k_{z}z)}$
\\

\noindent To simplify things here position variable is just `$z$', i.e we assume the wave is travelling in z direction and the polarized characteristics are seen in in the X-Y plane.
thus it is easier to figure out the effects of these polarized gravitational waves.

\pagebreak
